\chapter{网络层}

\section{网络层}

\subsection{数据平面与控制平面}

在网络中,每一台主机和路由器都有网络层。网络层提供了在发送主机与接收主机之间传送段的服务,在发送端将段封装到数据报中,在接收端将段上交给传输层实体。\\

网络层可以分解为数据平面(data plane)和控制平面(control plane)这两个互相作用的部分。数据平面指每台路由器的功能,决定从路由器的输入端口如何转发(forwarding)到输出端口。控制平面指的是数据包如何在路由器之间路由(routing),决定了数据报从源到目标主机之间的端到端路径。\\

每台路由器都有一个转发表(forwarding table),路由器使用分组首部的字段值在转发表中索引,指出该分组将被转发的输出端口。\\

配置路由器转发表的传统方法是,根据路由器中的路由选择算法与其它路由器中的路由选择算法通信,从而计算出它的转发表的值,这种通信根据路由选择协议交换包含路由选择信息的路由选择报文。\\

另一种方法称为软件定义网络SDN(Software-Defined Networking),通过由ISP或第三方管理的远程控制器计算并分发转发表以供每台路由器使用。路由器只执行转发,没有选择功能。因为计算转发表并与路由器交互的控制器是用软件实现的,故得其名。

\newpage

\section{IP协议}

\subsection{IPv4}

IPv4地址长度为32位(4字节),通常以点分十进制记法(dotted-decial notation)表示。例如11000001 00100000 11011000 00001001的点分十进制表示为192.32.216.9。\\

IP地址由网络号(subnet part)和主机号(host part)构成,网络号相同的IP地址属于同一网段。\\

国际标准组织ISO将IP地址分为5大类。\\

\begin{figure}[H]
    \centering
    \begin{tikzpicture}
        \draw (0,8) node[left, yshift=0.5cm]{A类} rectangle (10,9);
        \draw (0,6) node[left, yshift=0.5cm]{B类} rectangle (10,7);
        \draw (0,4) node[left, yshift=0.5cm]{C类} rectangle (10,5);
        \draw (0,2) node[left, yshift=0.5cm]{D类} rectangle (10,3);
        \draw (0,0) node[left, yshift=0.5cm]{E类} rectangle (10,1);

        \draw (0.5,8) -- (0.5,9);
        \draw (3,8) -- (3,9);
        \draw (0.25,8.5) node{1};
        \draw (1.75,8.5) node{网络号7位};
        \draw (6.5,8.5) node{主机号24位};

        \draw (1,6) -- (1,7);
        \draw (5,6) -- (5,7);
        \draw (0.5,6.5) node{10};
        \draw (3,6.5) node{网络号14位};
        \draw (7.5,6.5) node{主机号16位};

        \draw (1.5,4) -- (1.5,5);
        \draw (7.5,4) -- (7.5,5);
        \draw (0.75,4.5) node{110};
        \draw (4.5,4.5) node{网络号21位};
        \draw (8.75,4.5) node{主机号8位};

        \draw (2,2) -- (2,3);
        \draw (1,2.5) node{1110};

        \draw (2.5,0) -- (2.5,1);
        \draw (1.25,0.5) node{1110};
    \end{tikzpicture}
    \caption{IP地址分类}
\end{figure}

\vspace{0.5cm}

A类地址的网络有$ 2^7 - 2 = 126 $(全为0和1的地址保留),每个网络能容纳$ 2^{24} - 2 = 1677214 $个主机。\\

B类地址的网络有$ 2^{14} - 2 = 16384 $,每个网络能容纳$ 2^{16} - 2 = 65534 $个主机。\\

C类地址的网络有$ 2^{21} - 2 = 2097152 $,每个网络能容纳$ 2^8 - 2 = 254 $个主机。\\

D类地址被用于多点广播(multicast),多点广播地址用来一次寻址一组计算机。\\

E类地址为将来使用保留,仅作实验和开发用。\\

国际规定有一部分IP地址用于局域网,不在公网中使用。它们的范围是:

\begin{itemize}
    \item 10.0.0.0 $ \sim $ 10.255.255.255
    \item 172.16.0.0 $ \sim $ 172.31.255.255
    \item 192.168.0.0 $ \sim $ 192.168.255.255
\end{itemize}

\vspace{0.5cm}

\subsection{子网掩码(subnet mask)}

目前因特网的地址分配策略采用的是无类别域间路由选择CIDR(Classless InterDomain Routing),IP地址的形式为a.b.c.d/x,其中最高x位构成了IP地址的网络部分。\\

