\chapter{链路层}

\section{链路层}

\subsection{链路层}

数据链路层用于跨物理层在网段节点之间传输数据,通常指以太网、无线局域网等通信手段。数据链路层提供了在网络的两个实体之间传输数据的功能,并且提供了差错检测用于纠正物理层中发生的错误。\\

在数据链路层中,链路层地址有很多中不同的称谓,如LAN地址、物理地址或者MAC地址。因为MAC地址是最流行的术语,所以一般称呼链路层地址为MAC地址(Media Access Control Address)。\\

在每个网络层数据报在传输之前,几乎所有的链路层协议都会将数据报封装为帧(frame)。如果帧太大的话,数据链路层会将大帧拆分为一个个的小帧,小帧能够使传输控制和错误检测更加高效。\\

\subsection{MAC协议}

MAC协议规定了帧在链路上的传输规则,MAC层负责传输介质的流控制和多路复用,其中MAC 地址主要用于识别数据链路中互联的节点。\\

