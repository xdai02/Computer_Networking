\chapter{传输层}

\section{多路复用与多路分解}

\subsection{传输层}

传输层协议为运行在不同主机上的应用进程之间提供了逻辑通信。在发送端,运输层将应用进程的报文添加传输层首部形成传输层分组,称为报文段(segment),这个过程被称为多路复用(multiplexing)。在接收端,网络层从数据报中提取传输层报文段,并交付给传输层,传输层处理报文段,将数据交付给应用进程,这个过程被称为多路分解(demultiplexing)。\\

应用层可以使用UDP和TCP这两种截然不同的传输层协议。其中UDP提供了一种不可靠、无连接的服务,因此,UDP不能保证一个进程发送的数据能够完整无缺地到达目的进程。而TCP提供了一种可靠的、面向连接的服务,通过使用流量控制、序号、确认和定时器,TCP确保正确地、按序地将数据交付给接收进程。\\

\subsection{无连接的多路复用/多路分解}

