\chapter{网络}

\section{因特网}

\subsection{因特网(Internet)}

因特网是一个世界范围的计算机网络,它互联了遍及全世界数十亿的计算设备,所有这些设备都称为主机(host)或端系统(end system)。端系统通过通信链路(communication link)和分组交换机(packet switch)连接到一起,不同的链路能够以不同的速率传输数据,链路的传输速率(transmission rate)使用比特/秒(bps, bit/s)来度量。端系统通过因特网服务提供商(ISP, Internet Service Provider)接入因特网。\\

当一台端系统要向另一台端系统发送数据时,发送端将数据分组,发送到目的端系统,在那里进行组装。一个分组所经历的一系列通信链路和分组交换机称为路径(route / path)。分组交换类似于现实中的货物运输,在出发地将货物分开并装上多辆卡车,每辆卡车独立通过公路运输到目的地,最后在目的地卸货并重新组装。\\

\subsection{协议(Protocol)}

在两个人或两台设备之间进行通信时需要遵守一些协议,协议就是用于管理通信的一组规则。传输控制协议TCP (Transmission Control Protocol)和网际协议IP (Internet Protocol)是因特网中两个最为重要的协议,因特网的主要协议统称为TCP/IP。\\

因特网标准(Internet standard)是经过充分测试的规约,只要是与因特网打交道,就会用到它们,并要服从于它们。因特网标准由IETF (Internet Engineering Task Force)研发,IETF的标准文档称为RFC (Request For Comment),目的是解决因特网先驱者们面临的网络和协议问题。它们定义了TCP、IP、HTTP、SMTP等协议,目前已经有将近7000个RFC。\\

\subsection{分布式应用程序(Distributed Application)}

分布式应用程序涉及多个相互交换数据的端系统,例如即时通信、实时道路信息、视频会议、多人游戏等。分布式应用程序的核心问题在于一个端系统上的应用程序如何能够向运行在另一个端系统上的应用程序发送数据。\\

套接字接口(socket interface)规定了运行在一个端系统上的程序向运行在另一个端系统上的特定程序交付数据的方式。例如Alice要给Bob寄一封信,当然Alice不能只是写完这封信就把它丢出窗外。Alice需要把信放入信封,在信封上根据指定格式写上收信人的全名、地址和邮政编码,信封上贴上邮票,再将信封投入信箱中。Alice想要寄信就必须要遵守邮政服务制定的这一套规则。因此,发送数据的程序也必须遵守socket接口,才能向接收数据的程序发送数据。\\

